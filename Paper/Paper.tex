% -----------------------------------------------
% Template for ISMIR 2010
% (based on earlier ISMIR templates)
% -----------------------------------------------

\documentclass{article}
\usepackage{ismir2010,amsmath,cite}
\usepackage{graphicx}
\usepackage{url}
\usepackage{algorithm,algorithmic}


% Title.
% ------
\title{Large-scale harmonic patterns clustering}

% Single address
% To use with only one author or several with the same address
% ---------------
%\oneauthor
% {Names should be omitted for double-blind reviewing}
% {Affiliations should be omitted for double-blind reviewing}

% Two addresses
% --------------
%\twoauthors
%  {First author} {School \\ Department}
%  {Second author} {Company \\ Address}

% Three addresses
% --------------
\threeauthors
  {First author} {Affiliation1 \\ {\tt author1@ismir.edu}}
  {Second author} {\bf Retain these fake authors in\\\bf submission to preserve the formatting}
  {Third author} {Affiliation3 \\ {\tt author3@ismir.edu}}
% what order do we use? Thierry, Ron, Dan? is Juan on?


\begin{document}
%
\maketitle
%
\begin{abstract}
We present a method for discovering typical patterns in music.
Similar to the \textit{shingle} idea, but done on a larger scale through
Echo Nest API, we cluster harmonic patterns from 43K songs. Patterns are
made of ordered chroma features, and not of summarization like gaussian
mixtures coefficients or NMF weights. We explain
how to operate on such large datasets, we analyze the obained clusters,
and we discuss there use for retrieval and encoding. All code is made
available.
\end{abstract}
%
\section{Introduction}\label{sec:introduction}

This template includes all the information about formatting manuscripts for the ISMIR 2010. 
Please follow these guidelines to give the final proceedings a uniform look. 
If you have any questions, please contact the Conference Management.
This template can be downloaded from the ISMIR 2010 web site (http://ismir2010.ismir.net).


\section{Previous Work}\label{sec:prevwork}
The idea is closely related to the \textit{Shingles} described by
Slaney and Casey \cite{Casey2006,Casey2007,Casey2008}. LSH: \cite{E2LSH}.
Barrington et al. recently studied music texture using a set of ordered
chroma vectors \cite{Barrington2009a}. 
Nearest neighbor for music also includes \cite{Cano2004} and ...


\section{Data}\label{sec:data}

\subsection{Echo Nest features}
Echo Nest analyze API: \cite{EchoNest}.

\subsection{Cowbell Dataset}
We have $3,720,091$ non zero bars, a bar usually contains
$4$ beats.


\subsection{uspop2002}
uspop: \cite{uspop2002}.


\section{Algorithm}\label{sec:algo}

Vector quantization. Online learning.

\subsection{Codebook Learning}
We initialize the codebook by choosing $K$ random points from our dataset.


\begin{algorithm}
\caption{Pseudocode of Vector Quantization}
\label{alg:vq}
\begin{algorithmic}
\STATE$l$ learning rate
\STATE$\{P_n\}$ set of patches
\STATE$\{C_k\}$ codebook of $K$ codes
%\STATE $m \leftarrow b$
\REQUIRE $0 < l \leq 1$
\FOR{$nIters$}
\FOR{$p \in \{P_n\}$}
\STATE$c \leftarrow min_{c \in C_k} dist(p,c)$
\STATE$c \leftarrow c + (p - c) * l$
\ENDFOR
\ENDFOR
\RETURN $u$
\end{algorithmic}
\end{algorithm}




\section{Experiments}\label{sec:experiments}





\section{Conclusion}
We are awesome. Here's how:
\begin{itemize}
\item large-scale
\item smarter shingles, based on beats and bars
\item clusters good for encoding
\item clusters good for something else?
\item free online API, code available\footnote{code not released yet to preserve submission's anonymity}
\item we have lab t-shirts
\end{itemize}



\small
\section{Acknowledgements}
Thierry is NSERC graduate fellow, or some title like that.


%\begin{thebibliography}{citations}
%\bibitem{Someone:04} 
%X. Someone and Y. Someone:
%{\it Title of the Book},
%Editorial Acme, Utrecht, 2004.
%\end{thebibliography}

\bibliography{tbm_bib}





\end{document}
